\section{Generalized Linear Models}
We relax the assumption that $\mu$ is linear. Instead, we assume that g $\circ \mu$ is linear, for some function $g$:\\

$g(\mu (\mathbf x)) = \mathbf x^ T \beta$

The function $g$ is assumed to be known, and is referred to as the link function. It maps the domain of the dependent variable to the entire real Line.

it has to be strictly increasing,

it has to be continuously differentiable and

its range is all of $\mathbb{R}$

\subsection{Multivariate Linear Regression}
\textbf{\textit{[R]}}
\begin{simplebsl}
\\Setup: Y = X%*%Beta + e
\\Modeling Assumptions in Linear Regression:
\\1. Linear Relationship. 
\\2. X ~ Normal, e_i ~(idd) Normal(mu=0, sigma^2)
\\3. E[e_i*e_j] = 0, E[X*e_j] = 0
\\Beta_hat ~ Normal(Beta, sigma^2*solve(trans(X)%*%X))
\\sigma_hat^2 = trans(Y - X%*%Beta_hat)%*%(Y - X%*%Beta_hat)/(n - p)
\end{simplebsl}

\subsection{Theorems for Hypothesis Testing:}
\textbf{\textit{[R]}}
\begin{simplebsl}
\\1. n*pop_var/sigma^2 ~ Chi-square(n - p)
\\2. Cochran's theorem. 
\end{simplebsl}

\subsection{T Test:}
\textbf{\textit{[R]}}
\begin{simplebsl}
\\T_n = trans(u)
\\%*%(Beta_hat - Beta)
\\/sigma_hat
\\*sqrt(trans(u)%*%solve(trans(X)%*%X)%*%u)
\\T_n ~ T(n-p)
\end{simplebsl}

\subsection{Canonical Exponential Family:}
\textbf{\textit{[R]}}
\begin{simplebsl}
\\f_theta(y) = e^((y*theta-b(theta))/phi + c(y,phi)) Given phi is known. 
\\nabla_Beta\{b(X_i%*%Beta)\} = trans(X_i)%*%b'(X_i%*%Beta)
\\For MLE: trans(X)%*%Y = trans(X)%*%b'(X%*%Beta) => (Normal dist.) Beta = solve(trans(X)%*%X)%*%trans(X)%*%Y
\\E[Y] = mu = b'(theta) => Canonical Link: \{b'\}^\{-1\}(mu) = theta = X%*%Beta
\\Var(Y) = b''(theta)*phi (don't need to use this in practice)
\end{simplebsl}

\subsection{Properties:}
\textbf{\textit{[R]}}
\begin{simplebsl}
\\1. Canonical link function is strictly increasing. 
\\2. The log-likelihood l(theta) is strictly concave when phi > 0. 
\\Common Canonical Links (g(mu)): 
\\Norm | mu             | theta^2/2       >> (b(theta))
\\Pois | log(mu)        | e^theta
\\Bern | log(mu/(1-mu)) | log(1+e^theta)
\\Gamm | -1/mu          | -log(-theta)
\end{simplebsl}

\subsection{The Exponential Family}

A family of distribution $\, \{ \mathbf{P}_{{\boldsymbol \theta }}: {\boldsymbol \theta }\in \Theta \} ,\,$  where the parameter space $\Theta \subset \mathbb {R}^ k\,$ is -$k$ dimensional, is called a $k$-parameter exponential family on $\mathbb{R}^1$ if the pmf or pdf $\, f_{\boldsymbol \theta }:\mathbb {R}^ q\to \mathbb {R}\,$ of $\, \mathbf{P}_{{\boldsymbol \theta }}\,$ can be written in the form:\\

$\displaystyle  \displaystyle f_{\boldsymbol \theta }(\mathbf{y})=h(\mathbf{y})\, \exp \left({\boldsymbol \eta }({\boldsymbol \theta })\cdot \mathbf{T}(\mathbf{y})-B({\boldsymbol \theta })\right)\qquad \text {where } \\ \begin{cases}  {\boldsymbol \eta }({\boldsymbol \theta })=\begin{pmatrix} \eta _1({\boldsymbol \theta }) \hdots \eta _ k({\boldsymbol \theta })\end{pmatrix}& :\mathbb {R}^ t\to \mathbb {R}^{1 \times k} \\ \mathbf{T}(\mathbf{y})=\begin{pmatrix} T_1(\mathbf{y})\\ \vdots \\ T_ k(\mathbf{y})\end{pmatrix}& :\mathbb {R}^ q\to \mathbb {R}^{k \times 1} \\ B({\boldsymbol \theta })& :\mathbb {R}^ t\to \mathbb {R}\\ h(\mathbf{y})& :\mathbb {R}^ q\to \mathbb {R}.\\ \end{cases}$\\


if $k=1$ it reduces to:\\

$\displaystyle  \displaystyle f_\theta (y)=h(y)\, \exp \left(\eta (\theta ) T(y)-B(\theta )\right)$ \\
\textbf{\textit{[R]}}
\begin{simplebsl}
\\f_theta(y) = h(y)*e^(eta(theta)%*%T(y)-B(theta))
\\Ex. 1/gamma(a)*(a/mu)^a*y^(a-1)*e^(-a*y/mu) = 1/gamma(a)*(a/mu)^a*e^((a-1)*ln(y)-a*y/mu): eta(theta)=[a-1, -a/mu] | T(y)=[ln(y), y]' | B(theta)=-1/gamma(a)*(a/mu)^a | h(y)=1
\end{simplebsl}